\documentclass[a4paper,10pt]{article}

%A Few Useful Packages
\usepackage{marvosym}
\usepackage{fontspec} %for loading fonts
\usepackage{xunicode,xltxtra,url,parskip} %other packages for formatting
\usepackage[absolute]{textpos}
\RequirePackage{color,graphicx}
\usepackage[usenames,dvipsnames]{xcolor}
% an alternative to Layaureo can be ** \usepackage{fullpage} **
\usepackage[big]{layaureo} %better formatting of the A4 page
\usepackage{supertabular} %for Grades
\usepackage{titlesec} %custom \section
\usepackage{multicol} %multiple colums lists
\usepackage{fontawesome} %github icon
\usepackage{upgreek} % micro symbol
\usepackage{hyperref} % Setup hyperref package, and colours for links
\usepackage{enumitem} % itemize spacing

\definecolor{linkcolour}{rgb}{0,0.2,0.6}
\hypersetup{colorlinks,breaklinks,urlcolor=linkcolour, linkcolor=linkcolour}

% FONTS
\defaultfontfeatures{Mapping=tex-text}
\setmainfont[
	Path = configurations/fonts/,
	SmallCapsFont = Fontin-SmallCaps.otf,
	BoldFont = Fontin-Bold.otf,
	ItalicFont = Fontin-Italic.otf
]{Fontin.otf}
%%%

% SPACING
\titleformat{\section}{\Large\scshape\raggedright}{}{0em}{}[\titlerule]
\titlespacing{\section}{0pt}{3pt}{3pt}
\addtolength{\voffset}{-1.3cm} %Tweak a bit the top margin

\setlength{\TPHorizModule}{30mm}
\setlength{\TPVertModule}{\TPHorizModule}
\textblockorigin{2mm}{0.65\paperheight}
\setlength{\parindent}{0pt}

\setlist{nolistsep, itemsep=1mm}


\begin{document}

\pagestyle{empty} % non-numbered pages

%--------------------TITLE-------------
\par{\centering{\Huge
	Cristóvão Diniz Trevisan
}\bigskip\par}

%--------------------SECTIONS-----------------------------------
\section{Dados Pessoais}
\begin{tabular}{rl}
	\textsc{Data de Nascimento:} & 03 Agosto 1993 \\
	\textsc{Endereço:}	& Santa Felicidade, Curitiba - PR, Brasil \\
	\textsc{Celular:}	& +55 (41) 98802-6262\\
	\textsc{GitHub:}	& \href{https://github.com/cristovao-trevisan}{\faGithub\ cristovao-trevisan} \\
	\textsc{Email:}	& \href{mailto:cristovao.trevisan@gmail.com}{cristovao.trevisan@gmail.com} \\
\end{tabular}

\section{Descrição Pessoal}
% Comentário a seguir salvo para felicidade futura
% Lorem Ipsum Sou Foda Lorem Ipsum

Obcecado naturalmente por programação e atualmente especializado em fullstack,
com conhecimento gerais em assuntos relacionados necessários (como segurança digital,
OOP, programação funcional e SASS) bem presentes. Sempre tentando aprender coisas
novas (atualmente C++ avançado e OpenGL). Com outra fissura (igualmente grande, senão
maior) por música. Atualmente procurando algum lugar com programadores bem melhores
que eu para extrair o máximo e eventualmente os superar.

\section{Linguagens De programação}
\begin{tabular}{rl}
	Iniciado:& php, Lisp, ML, Assembly, Python\\
	Padawan:& C, Ruby (on Rails), C\#, Objective-C, Makefile\\
	Cavalheiro Jedi:& Java (Swing e Android), Bash, SQL(mySQL, Postgres, \underline{ORM's})\\
	Mestre Jedi:& C++14\\
	Grande Mestre:& JavaScript
\end{tabular}

\section{Frameworks e Paradigmas de Javascript}
\begin{multicols}{4}
	\begin{itemize}
		\item React
		\item React-Native
		\item Redux
		\item Cordova
		\item Express
		\item NW.js
		\item Node.js (C++)
		\item Webpack
	\end{itemize}
\end{multicols}

\section{Línguas Extrangeiras}
\begin{tabular}{rl}
	\textsc{Inglês:}&Fluente\\
	\textsc{Espanhol:}&Básico\\
\end{tabular}

\section{Experiência de Trabalho}
\begin{tabular}{r|p{11cm}}
	\emph{Atual} & Desenvolvedor Pleno na Startup Mindtech Tecnologias \\
	\textsc{Outubro 2016} &	\footnotesize{Desenvolvimento completo de plataforma IOT
	para controle	remoto (inicialmente de portões, posteriormente para alarmes e ainda
	genérico). Como único responsável por todo o desenvolvimente foram criados desde os
	layouts	das placas, para o firmware ($\upmu$C ESP8266), para o servidor e finalmente
	aplicativo híbrido. Algumas tecnologias utilizadas foram: C++, Node.js, Postgres,
	Cordova, React, React-Native. \href{http://mindtech.com.br/}{Site do produto}}\\
	\multicolumn{2}{c}{} \\

	\emph{Julho 2014} & Desenvolvedor Júnior em Laboratório da UTFPR\\
	\textsc{Agosto 2012} & \footnotesize{Desenvolvimento de sistema para monitoramento
	sem fio de linhas de energia em tempo real utilizando rede mesh. Foi criada uma
	solução para monitorar a tensão em transformadores e verificar o estado assim
	como histórico de medidas em ambiente online. Algumas das tecnologias: C++ (Arduino),
	ZigBee,	Ruby on Rails, Cassandra (NoSQL)}\\
	\multicolumn{2}{c}{} \\
\end{tabular}

\section{Educação Formal}
\begin{tabular}{rl}
	\textsc{Atual} & Engenharia Eletrônica - \textbf{UTFPR}\\
	\textsc{Jan. 2011}& \small TCC: Digitalizador de Guitarra. Rendimento: 0.7639/1\\
	\\
	\textsc{Ago. 2015} & Computer Engineering - \textbf{Syracuse University}, NY - USA\\
	\textsc{Ago. 2014} & \small Intercâmbio focado em programação. GPA: 3.481/4\\
\end{tabular}

\section{Menções Interessantes mas Não Necessariamente Relevantes}
\begin{itemize}
	\item 3 participações em maratonas de programação
	\item 2 participações em hackathons
	\item Estudo em violão e música erudita
	\item Atualmente entretido em um curso avançado de teatro musical
	\item Professor de violão em Comunidade Escola
	\item Há 11 anos jogador de Dota
\end{itemize}


\end{document}
